\documentclass{bmstu}

\usepackage{pdfpages}

\addbibresource{inc/biblio/sources.bib}

\begin{document}

\includepdf[pages=-]{inc/img/title.pdf}

{\centering \chapter*{РЕФЕРАТ}}

Отчет X с., X рис., X табл., X источн., X прил.

\noindent ВИДЕО, ВИДЕОПОТОК, ВИДЕОИЗОБРАЖЕНИЕ, РАЗРЕШЕНИЕ, КАЧЕСТВО, НЕЙРОННЫЕ СЕТИ

Объектом исследования является X.

Цель работы: классификация существующих алгоритмов X.

В результате исследования было проведено сравнение X по X критериям.

Область применения результатов --- выбор алгоритма X.

Результат работы.

\maketableofcontents

{\centering \chapter*{ПЕРЕЧЕНЬ СОКРАЩЕНИЙ И ОБОЗНАЧЕНИЙ}}

В настоящем отчете о НИР применяют следующие сокращения и обозначения:

\begin{table}[H]
\begin{tabular}{p{3cm}p{13.5cm}}
VSR & Суперразрешение видео (video super-resolution)
\tabularnewline
\end{tabular}
\end{table}

{\centering \chapter*{ВВЕДЕНИЕ}}
\addcontentsline{toc}{chapter}{ВВЕДЕНИЕ}

Суперразрешение --- это способ получения изображения или видеоизображения с высоким разрешением из изображений с низким разрешением~\cite{Park2003}. 
Суперразрешение обеспечивает изображение или видео повышенного качества с более подробной информацией о сцене, что важно для точного анализа~\cite{Daithankar2021}. 
В отличие от суперразрешения одного изображения (single-image super-resolution), основная цель суперразрешения видео --- не только восстановить больше мелких деталей при сохранении крупных, но и сохранить согласованность движения.

Во многих областях, работающих с видео, люди имеют дело с различными типами деградации видео деградации, включая понижение разрешения. 
Разрешение видео может снизиться из-за несовершенства измерительных устройств. 
Плохое освещение и погодные условия добавляют шум. 
Движение объектов и камеры также ухудшает качество видео. 
Методы суперразрешения помогают восстановить исходное видео. 
Это полезно в широком спектре приложений, таких как:
\begin{enumerate}
\item[1)] видеонаблюдение (для улучшения качества видео, снятого с камеры, а также распознавания номеров автомобилей и лиц);
\item[2)] медицинская визуализация (чтобы лучше обнаружить некоторые органы или ткани для клинического анализа и медицинского вмешательства);
\item[3)] судебно-медицинская экспертиза (для помощи в расследовании в ходе уголовного процесса);
\item[4)] астрономия (для улучшения качества видео звезд и планет);
\item[5)] дистанционное зондирование (для облегчения наблюдения за объектом);
\item[6)] микроскопия (для усиления возможностей микроскопов).
\end{enumerate}

Суперразрешение видео также помогает решить задачу обнаружения объектов, распознавания лиц и символов (в качестве этапа предварительной обработки).

Существует множество подходов к решению этой задачи, но она по-прежнему остается популярной и сложной.

Цель научно-исследовательской работы: провести обзор существующих алгоритмов увеличения разрешения видеопотока и классифицировать их по сформулированным критериям.

Задачи научно-исследовательской работы:
\begin{enumerate}
\item[1)] исследовать предметную область увеличения разрешения видеопотока;
\item[2)] проанализировать известные алгоритмы увеличения разрешения видеопотока;
\item[3)] сформулировать критерии для сравнения этих методов;
\item[4)] сравнить алгоритмы увеличения разрешения видеопотока по сформулированным критериям.
\end{enumerate}

\chapter{Анализ предметной области}

\section{Математическое объяснение}

Как правило, исследователи рассматривают процесс деградации кадров как
\begin{equation}
\{y\} = (\{x\} * k) \downarrow _{s} + \{n\},
\end{equation}
где $\{x\}$ --- исходная последовательность кадров высокого разрешения, $k$ --- ядро размытия, $*$ --- операция свертки, $\downarrow _{s}$ --- операция уменьшения масштаба, $\{n\}$ --- аддитивный шум, $\{y\}$ --- последовательность кадров низкого разрешения.

Суперразрешение --- это обратная операция, поэтому ее задача состоит в том, чтобы оценить последовательность кадров $\{\overline{x}\}$ по последовательности кадров $\{y\}$ так, чтобы $\{\overline{x}\}$ близко к исходному $\{x\}$. 
Ядро размытия, операция уменьшения масштаба и аддитивный шум должны быть оценены для заданных входных данных для достижения лучших результатов.

Подходы суперразрешения, как правило, содержат больше компонентов, чем аналоги изображений, поскольку им необходимо использовать дополнительное временное измерение. 
Сложные конструкции --- не редкость. 
Некоторые наиболее важные компоненты суперразрешения видео управляются четырьмя основными функциями: распространение, выравнивание, агрегирование и увеличение разрешения.

При работе с видео временная информация может использоваться для улучшения качества масштабирования. 
Можно также использовать методы суперразрешения одиночного изображения, генерирующие кадры с высоким разрешением независимо от их соседей, но это менее эффективно и приводит к временной нестабильности. 
Существует несколько традиционных методов, которые рассматривают задачу суперразрешения видео как задачу оптимизации.  
В последние годы методы масштабирования видео, основанные на глубоком обучении, превосходят традиционные.

\chapter{Традиционные алгоритмы увеличения разрешения видеопотока}

Существует несколько традиционных методов масштабирования видео. 
Эти методы пытаются использовать некоторые естественные настройки и эффективно оценивать движение между кадрами. 
Кадр высокого разрешения восстанавливается на основе как естественных настроек, так и предполагаемого движения.

Различные методы, связанные с суперразрешением, делятся на частотные и пространственные~\cite{Daithankar2021}.

Суперразрешение может быть оптическим или геометрическим. 
В оптических методах используются характеристики оптики, датчиков и компонентов дисплея устройства визуализации, которые отвечают за ухудшение качества или разрешения изображения. 

\section{Частотный домен}

\section{Пространственный домен}

\subsection{Iterative back-projection methods}

\subsection{Iterative adaptive filtering algorithms}

\subsection{Direct methods}

\subsection{Non-parametric algorithms}

\subsection{Probabilistic methods}

\chapter{Алгоритмы увеличения разрешения видеопотока, основанные на глубоком обучении}

\subsection{Aligned by motion estimation and motion compensation}

\subsection{Aligned by homography}

\subsection{Spatial non-aligned}

\subsection{3D convolutions}

\subsection{Recurrent neural networks}

\subsection{Videos}

\chapter{Классификация алгоритмов увеличения разрешения видеопотока}

\section{Критерии оценки алгоритмов увеличения разрешения видеопотока}

\section{Сравнение алгоритмов увеличения разрешения видеопотока}

Приведенную выше информацию можно записать в таблицу~\ref{tabular:comparison}. 

\begin{table}[H]
\caption{Сравнение алгоритмов блокчейн-консенсуса}
\label{tabular:comparison}
\begin{tabular}{|p{3cm}|p{2cm}|p{2.4cm}|p{2.cm}|p{2.cm}|p{2.cm}|}
\hline
\textbf{Критерий} & \textbf{PoW} & \textbf{PoS} & \textbf{HC} & \textbf{PoC} & \textbf{PoI}
\tabularnewline
\hline
Среднее время создания блока, с & 12--600 & 4.5--60 & 300 & 240 & 60
\tabularnewline
\hline
Стойкость к двойному расходованию, \% & 51 & 33 или 51 & 51 & 50 & 51
\tabularnewline
\hline
Количество транзакций в секунду & 7--500 & 173--1000 & 14 & 80 & 4000
\tabularnewline
\hline
\end{tabular}
\end{table}

{\centering \chapter*{ЗАКЛЮЧЕНИЕ}}
\addcontentsline{toc}{chapter}{ЗАКЛЮЧЕНИЕ}

В ходе выполнения научно-исследовательской работы была достигнута поставленная цель, а также решены все задачи:
\begin{enumerate}
\item[1)] исследована предметную область увеличения разрешения видеопотока;
\item[2)] проанализированы известные алгоритмы увеличения разрешения видеопотока;
\item[3)] сформулированы критерии для сравнения этих методов;
\item[4)] проведено сравнение алгоритмов увеличения разрешения видеопотока по сформулированным критериям.
\end{enumerate}

{\centering {\center\printbibliography[title=СПИСОК ИСПОЛЬЗОВАННЫХ ИСТОЧНИКОВ]}}
\addcontentsline{toc}{chapter}{СПИСОК ИСПОЛЬЗОВАННЫХ ИСТОЧНИКОВ}

{\centering \chapter*{ПРИЛОЖЕНИЕ А}}
\addcontentsline{toc}{chapter}{ПРИЛОЖЕНИЕ А Презентация}

\end{document}
